\documentclass[12pt]{article}

\usepackage{amsmath}

\usepackage{graphicx}

\usepackage{hyperref}

\usepackage[utf8]{inputenc}

% Redefine the section command to include a dot after the number
\renewcommand\thesection{\arabic{section}.}
\renewcommand\thesubsection{\thesection\arabic{subsection}.}
\renewcommand\thesubsubsection{\thesubsection\arabic{subsubsection}.}

\title{ME/EE/CS 133a Homework 1}

\author{Baaqer Farhat}

\date{10–07–24}

\begin{document}

\maketitle

\section{Problem 1}

\begin{itemize}

\item (a) 

There exists 5 degrees of freedom. A Rigid Body in 3D space has 6 Degrees of Freedom bieng (x, y, z) and (pitch, yaw, roll).
A line segment has no thickness so its incapable of achieving the roll rotational DOF. A line segment is 2 dimensional. 

\item (b)

There exists 2 degrees of freedom. A Rigid Body in 3D space has 6 Degrees of Freedom bieng (x, y, z) and (pitch, yaw, roll). A 
torus is a 2 dimensional manifold. It can only rotate vertically (pitch) and horizantally (yaw). 

\item (c)

5 Degrees of freedom.

\item (d)

6 degrees of freedom.

\end{itemize}

\section{Problem 2}

\begin{itemize}
    \item (a) and (b)

    For both Scenarios there are 3 Degrees of Freedom. Both cars (controlled by one infront) can move along (X, Y, $\Theta$). The 
    car can travel horizantally, then rotate around its axis then move along vertically.

\end{itemize}

\section{Problem 3}

\begin{itemize}
    \item (a)
    There Exists 5 Degrees of Freedoms on the human arm excluding the wrist and hand. Shoulder joint has 3 degrees of freedom bieng veritocal, horizantal and rolling movement. 
    The elbow can rotate horizantally and roll. 
\end{itemize}



\end{document}